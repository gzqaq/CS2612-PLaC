\documentclass[12pt]{article}

\usepackage{geometry}
\geometry{a4paper}

\usepackage{graphicx}
\usepackage{caption}

\usepackage{ctex}

\usepackage{amsmath}
\usepackage{amsfonts}
\usepackage{amsthm}
\usepackage{stmaryrd}

\usepackage{listings}
\usepackage{xcolor}

\definecolor{gray}{rgb}{0.5,0.5,0.5}

\lstset{
  frame=single,
  framerule=0pt,
  backgroundcolor=\color{gray!15},
  xleftmargin=2em,
}

\newcommand{\leqab}{\leq_{A \to B}}
\newcommand{\leqb}{\leq_B}
\newcommand{\st}{\textrm{ s.t. }}
\newcommand{\lub}{\textrm{lub}}


\title{Assignment 1104}
\author{Ziqin Gong $\quad$ 520030910216}
\date{}

\begin{document}
  \maketitle

  \section*{1}

    \subsection*{(1)}

      \begin{proof}
        1) 自反性:任意$a\in A$,由$(B,\leqb)$的自反性可得$f(a)\leqb f(a)$,即$f\leqab f$.

        2) 传递性:若$f \leqab g,\ g \leqab h$,由定义可得:
           \begin{equation}
           \begin{cases}
             \forall a, f(a) \leqb g(a) \\
             \forall b, g(b) \leqb h(b)
           \end{cases}
           \end{equation}
           故任意$a \in A$,$f(a) \leqb g(a),\ g(a) \leqb h(a)$。由$(B,\leqb)$的传递性可得:
           \begin{gather}
             \forall a \in A, f(a) \leqb h(a)
           \end{gather}
           即$f \leqab h$。

        3) 反对称性:若$f \leqab g,\ g \leqab f$,由定义可得:
           \begin{equation}
           \begin{cases}
             \forall a, f(a) \leqb g(a) \\
             \forall b, g(b) \leqb f(b)
           \end{cases}
           \end{equation}
           故任意$a \in A$,$f(a) \leqb g(a),\ g(a) \leqb f(a)$。由$(B,\leqb)$的反对称性可得:
           \begin{gather}
             \forall a \in A, f(a) = g(a)
           \end{gather}
           即$f \leqab g$。

        综上,$(A \to B, \leqab)$是一个偏序集。
    \end{proof}

  \subsection*{(2)}

    \begin{proof}
      已证$(A \to B, \leqab)$为偏序集,下面证其完备性。

      取任意链$C=\{f_1,f_2,\cdots\} \st f_1 \leqab f_2 \leqab \cdots$,定义如下函数:
      \begin{gather}
        f_\infty(x) = \lub(\{f_1(x), f_2(x), \cdots\})
      \end{gather}
      则$f_\infty$为该链$C$的上确界。证明如下:

      1) 任取$f_i \in C$,则对任意$a \in A$,$f_i(a)$属于$(B,\leqb)$中的链$C_B=\{f_1(a),f_2(a),\cdots\}$。
         故由$(B,\leqb)$的完备性可得:
         \begin{gather}
           \forall a \in A, f_i(a) \leqb \lub(\{f_1(a),f_2(a),\cdots) = f_\infty(a) \\
           \Leftrightarrow f_i \leqab f_\infty
         \end{gather}
         故任意$f_i \in C$,$f_i \leqab f_\infty$。

      2) 对于某个$g \in A \to B \st \forall f_i \in C, f_i \leqab g$,即对任意$a \in A$,$f_i(a) \leqb g(a)$。
         又因为$f_i(a) \in \{f_1(a),f_2(a),\cdots\}$这条$(B,\leqb)$上的链,故由其完备性:
         \begin{gather}
           \forall a \in A, \lub(\{f_1(a),f_2(a),\cdots\}) \leqb g(a) \\
           \Leftrightarrow \forall a \in A, f_\infty(a) \leqb g(a) \\
           \Leftrightarrow f_\infty \leqab g
         \end{gather}
         故任意$g \in A \to B \st \forall f_i \in C, f_i \leqab g$,都有$f_\infty \leqab g$。
    \end{proof}

  \subsection*{(3)}

    \begin{proof}
      已证$(A \to B, \leqab)$为偏序集,下面证其为完备格。

      取任意子集$S=\{f_1,f_2,\cdots\}$,定义如下函数:
      \begin{gather}
        f_\infty(x) = \lub(\{f_1(x), f_2(x), \cdots\})
      \end{gather}
      则$f_\infty$为该子集$S$的上确界。证明如下:

      1) 任取$f_i \in S$,则对任意$a \in A$,$f_i(a)$属于$(B,\leqb)$的子集$S_B=\{f_1(a),f_2(a),\cdots\}$。
         故由$(B,\leqb)$为完备格可得:
         \begin{gather}
           \forall a \in A, f_i(a) \leqb \lub(\{f_1(a),f_2(a),\cdots) = f_\infty(a) \\
           \Leftrightarrow f_i \leqab f_\infty
         \end{gather}
         故任意$f_i \in S$,$f_i \leqab f_\infty$。

      2) 对于某个$g \in A \to B \st \forall f_i \in S, f_i \leqab g$,即对任意$a \in A$,$f_i(a) \leqb g(a)$。
         又因为$f_i(a) \in \{f_1(a),f_2(a),\cdots\}$这个$(B,\leqb)$的子集,故由其为完备格:
         \begin{gather}
           \forall a \in A, \lub(\{f_1(a),f_2(a),\cdots\}) \leqb g(a) \\
           \Leftrightarrow \forall a \in A, f_\infty(a) \leqb g(a) \\
           \Leftrightarrow f_\infty \leqab g
         \end{gather}
         故任意$g \in A \to B \st \forall f_i \in S, f_i \leqab g$,都有$f_\infty \leqab g$。
    \end{proof}

  \section*{2}

    \subsection*{(1)}

      \begin{proof}
        1) 单调性:已知函数$f(X)=X$单调,而$\circ$可保持$\supseteq$的关系,故$F(X)$单调。

        2) 不连续:取如下无穷链:
           \begin{gather}
             \{0,1,2,3,\cdots\} \supseteq \{1,2,3,\cdots\} \supseteq \{2,3,\cdots\} \supseteq \cdots
           \end{gather}
           其$\lub$为$\emptyset$,则$F(\lub(C))=F(\emptyset)=\emptyset$。
           对每个元素做$F$映射可得如下集合:
           \begin{gather}
             \left\{\{-1,1,2,3,\cdots\}, \{-1,2,3,\cdots\}, \{-1,3,\cdots\}, \cdots \right\}
           \end{gather}
           其$\lub$为$\{-1\} \neq \emptyset$。故不连续。
      \end{proof}

    \subsection*{(2)}

      $\emptyset$

    \subsection*{(3)}

      \begin{align}
        F(\mathbb{Z}) &= \{-1,1,2,3,\cdots\} \\
        F^{(2)}(\mathbb{Z}) &= \{-1,2,3,\cdots\} \\
        F^{(3)}(\mathbb{Z}) &= \{-1,3,\cdots\} \\
        \cdots \\
        F^{(n)}(\mathbb{Z}) &= \{-1,n,\cdots\}
      \end{align}
      故
      \begin{gather}
        \bigcap_{n\in\mathbb{N}}\left(F^{(n)}(\mathbb{Z})\right) = \{-1\}
      \end{gather}
      而
      \begin{gather}
        F(\{-1\}) = \emptyset
      \end{gather}
      故不是不动点。

\end{document}
