\documentclass[12pt]{article}

\usepackage{geometry}
\geometry{a4paper}

\usepackage{graphicx}
\usepackage{caption}

\usepackage{ctex}

\usepackage{amsmath}
\usepackage{amsthm}

\usepackage{listings}
\usepackage{xcolor}

\definecolor{gray}{rgb}{0.5,0.5,0.5}

\lstset{
  frame=single,
  framerule=0pt,
  backgroundcolor=\color{gray!15},
  xleftmargin=2em,
}


\title{Assignment 1018}
\author{Ziqin Gong $\quad$ 520030910216}
\date{}

\begin{document}
  \maketitle

  \section*{1}

    \subsection{判断题1}
    
      是。

    \subsection{判断题2}

      是。

    \subsection{证明or否定}

      \begin{proof}

        1) 显然当$e = n \textrm{ or } x$时,命题成立

        2) 假设$\textrm{CF}(e_1) \equiv e_1, \textrm{CF}(e_2) \equiv e_2$.

           若$e=e_1+e_2$.

             若$e_1,e_2$都是常数,则显然成立。

             否则,$\textrm{CF}(e_1+e_2) = \textrm{CF}(e_1)+\textrm{CF}(e_2) \equiv e_1 + e_2 = e$,成立。

          类似地,对于$e=e_1 - e_2, e = e_1 * e_2$命题都成立。

        综上,对任意表达式做常量折叠变换都保持语义等价。
      \end{proof}

    \subsection{能否用于编译优化}

      可以。由于对任意表达式做常量折叠变换都不会改变其语义值,从而能够达到简化AST的效果。


\end{document}
